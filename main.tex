\documentclass[10pt]{article}
\usepackage{color}
\usepackage{xcolor}
\usepackage{url}
\usepackage{ulem}
\usepackage{mdwlist}
\usepackage{booktabs}
\usepackage{my-stdpkgs}
\usepackage{my-typography}
%\usepackage[numbers,sort&compress]{natbib}


%\usepackage{hyperref}
\usepackage{amsmath}
\usepackage{float}
\usepackage{amsfonts}
\usepackage{morefloats}
\setcounter{page}{1}
\pagenumbering{arabic}

\makeatletter
\newcommand{\ccell}[3][]{%
  \kern-\fboxsep
  \if\relax\detokenize{#1}\relax
    \expandafter\@firstoftwo
  \else
    \expandafter\@secondoftwo
  \fi
  {\colorbox{#2}}%
  {\colorbox[#1]{#2}}%
  {#3}\kern-\fboxsep
}
\makeatother
\definecolor{cellgray}{gray}{0.9}

\newcommand{\yesy}{\ccell{green}{Y}}
\newcommand{\non}{\ccell{red}{N}}


%TODO
%%%%%%%%%%%%%%%%%%%%%%%%%%%%%%%%%%%%%%%%%%%%%%%%%%%%%%%%%%%%%%%%%%%%%%%%
%Discuss title
%Do we discuss dense vs. sparse for each case or have a separate section for sparse?
%Should we consider graphical models?
%Should abstract emphasize bringing TD's to CS community, or emphasize computational limits?
%Move to bitbucket
%By next meeting make a grid of software :  latex booktabs



%Tentative title
\title{Data Analysis using Tensor Decompositions: Challenges and Applications (TODO)}

%Tentative author list
\author{
  Casey Battaglino$^\ast$, Joachim Perros$^\ast$, Jiajia Li$^\ast$, \\
   Jimeng Sun$^\ast$, Richard Vuduc$^\ast$
  \\ Georgia Institute of Technology, Atlanta, GA
  \\ $^\ast$ School of Computational Science and Engineering \\
  \{cbattaglino3,perros,jiajiali,jsun,richie\}@gatech.edu } \date{}


%%%%%%%%%%%%%%%%%%%%%%%%%%%%%%%%%%%%%%%%%%%%%%%%%%%%%%%%%%%%%%%%%%%%%%%%
\begin{document}
%%%%%%%%%%%%%%%%%%%%%%%%%%%%%%%%%%%%%%%%%%%%%%%%%%%%%%%%%%%%%%%%%%%%%%%%

\maketitle

\begin{abstract}
Linear algebra underpins the majority of data analysis techniques, but multi-linear algebra remains a fringe topic despite its natural applicability to the high-dimensionality that arises in a Big Data setting. Tensor methods have been the subject of much research in signal processing, physics, and applied mathematics, but haven't seen wide adoption in the wider computer science community. One particularly difficult and relevant issue is the efficient low-rank representation of highly sparse, high-dimensional data.

This survey has two goals: to demistify tensor approaches in data science (by moving away from clunky traditional notation), and to clearly delineate where current approaches (both dense and sparse) become computationally infeasible as a call towards future research in scalability.


\end{abstract}

\tableofcontents
%=======================================================================
\section{Introduction} \label{sec:intro}

%=======================================================================
\section{Background} \label{sec:background}
\subsection{Definitions} \label{sec:definitions}
\subsubsection{Tensor Representation}
\subsubsection{Multilinear Rank}
Computing rank is NP complete, and best-rank approximation is ill-posed~\cite{Kolda09tensordecompositions}
\subsubsection{Tensor Algebra}
\subsubsection{Tensor Contractions}
\subsubsection{Tensor Network Diagrams}

\section{Data Science Applications}
In signal processing we may have both temporal and spatial components to our signal (in combination with frequency)

Importance of multi-linear rank-reduction.

Feature selection, feature extraction, classification, multi-way clustering.
Blind-source separation. 
psycho-metrics, chemometrics, signal processing, numerical linear algebra, computer vision, watermarking, numerical analysis, data mining, neuroscience, graph analysis, ICA.
quantum chemistry(Hartree-Fock, MD), EM wave scattering/logging. 3-dimensional PDEs. 
%=======================================================================
\section{Low-Rank Approximations}
%mention need for compression

Tucker, CP are old concepts. Decomposition into block terms~\cite{Lathauwer08decompositionsof} generalizes standard methods.

\subsection{CP Decomposition}
Introduced:~\cite{hitchcock-sum-1927}
CP breaks curse of dimensionality: instead of $n^d$ parameters we need only $dnr$.
CP is ill-posed, different regularizations can be used.
CP assumes every tensor mode is involved with every `concept.' Restrictive for noisy higher order data (-Joachim)
\subsection{Tucker Decomposition}
Introduced:~\cite{Tuck1966c}
Instead of $n^d$ parameters we need only $\prod_{i=1}^d r_i + n\sum_{i=1}^d r_i$.
Stable, we can use SVD~\cite{Lathauwer00amultilinear}.
We can determine $r_1, \dots, r_d$ as a function of $\epsilon$.

If the Tucker Decomposition is properly normalized, it is also considered an HOSVD.

Exponential storage, time. In sparse case, we have same problem: dense results (-Joachim)
\subsection{Hierarchical Tucker}
Introduced:~\cite{Grasedyck:2010:HSV:1958286.1958311}
B's can express a linear combination of vectors in the U's below them.
Obst 1: SVD of tensor unfoldings
Obst 2: Intermediate memory blowup (computing core tensor)... dense intermediate result
\subsubsection{Sparse H-Tucker}
\subsection{Tensor Train}
Introduced:~\cite{Oseledets:2011:TD:2079141.2079149}
%=======================================================================
\section{Computational Challenges}
\subsection{Curse of Dimensionality}
\subsubsection{Time}
\subsubsection{Memory Blowup}
\subsubsection{Storage}
\subsection{Memory Efficiency}
\subsubsection{Representation Issues}
\subsubsection{Distributed Memory Issues}
\subsubsection{Communication Bounds}
\subsection{Sparsity}
\subsection{Stability}
%=======================================================================
%Section on: common operations? computational bottlenecks?
\section{Experiments}
\subsection{Breaking Points}
\subsubsection{CP Decomposition}
\subsubsection{Tucker/HOSVD}
\subsubsection{Hierarchical Tucker}
\subsubsection{Tensor Train}
\subsection{Comparisons}
%=======================================================================
\section{Current Implementations}
\subsection{Tensor Toolbox}

Basic tensor operations mainly consist of tensors and matrices conversion (matricization and tensorization) and different kinds of tensor multiplications. A bunch of libraries are built to implement part or all of basic tensor operations, like FTensor~\cite{Landry:2003:IHP:1240120.1240122,FTensor}, LTensor~\cite{LTensor}, Blitz++~\cite{blitz}, Boost Multidimensional Array Library~\cite{boost-multiarray}, the Tensor C++ library~\cite{tensorCPP}, and HUJI Tensor Library~\cite{huji}. All the libraries are written in C++ and are sequential ones, but they target on various areas. Besides, Tensor Contraction Engine(TCE)~\cite{TCE} and Cyclops Tensor Framework(CTF)~\cite{CTF} also realize basic tensor operations, especially tensor contraction. Both the two tools can be used in a distributed memory system by optimizing communication cost. CTF also supports two-level parallelism (MPI+OpenMP).

As concerned the low-order tensor family, tensor decomposition is a common method. 
One of the most popular MATLAB packages is the
Tensor Toolbox~\cite{TensorToolbox} which supports CP and Tucker decomposition for dense, sparse, and structured tensors. It also supports scalable tensor decomposition proposed by T. Kolda and J. Sun~\cite{Kolda:2008:STD:1510528.1511376}. Other tools also support some (different) tensor decomposition methods, e.g. N-way Toolbox~\cite{Nway-Paper,Nway}, CuBatch~\cite{CuBatch}, PLS\_Toolbox~\cite{PLS-toolbox}, Tensorlab~\cite{Tensorlab}, TDALAB~\cite{TDALAB,TDALAB_online}, TENSORBOX~\cite{TENSORBOX}, and PARCUBE~\cite{PARCUBE}. They are all built on MATLAB, and the last three tools are based on Tensor Toolbox. Other than Tensor Toolbox, Tensorlab and PARCUBE also support sparse input tensors. All the tensor decomposition tools are sequential.

For high-order tensors in machine learning area, tensor networks methods are widely used. Tensor Train Toolbox~\cite{tt-toolbox} is for tensor train method, while Hierarchical Tucker Toolbox~\cite{HT,Kressner:2014:A9H:2610268.2538688} focuses on Hierarchical Tucker method. Both of them are built on MATLAB. TensorCalculus Library~\cite{Calculus} is for more general tensor networks, supporting both tensor train and hierarchical Tucker. Tensor Network Theory Library~\cite{TNT} is another kind of tensor networks, based on tensor network theory. TensorCalculus Library and Tensor Network Theory Library are both implemented in C++. For now, all the four tools are sequential. 

There are some software based on density-matrix renormalization group (DMRG) method to do decomposition, including the Heidelberg MCTDH Package (Fortran)~\cite{MCTDH}, Intelligent Tensor (ITensor, C++)~\cite{ITensor}, ALPS(C++)~\cite{ALPS}, and Block(C++)~\cite{Block}. 


\begin{center}
    \begin{tabular}{ | l | l | l | l | l | l | l | l |}
    \hline
    Package 					
    & Language & $\underline{A} \times \underline{B}$ & $\underline{A} \otimes \underline{B}$ & CP & Tucker & TN & Domain \\ \hline
    scikit-tensor					
    & Python & \yesy & \yesy & \yesy & \yesy & \non & Physics \\ \hline
    PyTensor				
    & Python & \yesy & \yesy & \non & \non & \non & Physics \\ \hline
    FTensor~\cite{Landry:2003:IHP:1240120.1240122,FTensor}					
    & C++ & \yesy & \yesy & \non & \non & \non & Physics \\ \hline
    LTensor~\cite{LTensor} 		
    & C++ & \yesy & \yesy & \yesy & \yesy & \non & Physics \\ \hline
    tensor (Theano) 			
    & Python & \yesy & \yesy & \yesy & \yesy & \non & ML \\ \hline
    Blitz++~\cite{blitz} 
    & C++ & \yesy & \yesy & \yesy & \yesy & \non & General \\ \hline
    Boost MDA~\cite{boost-multiarray} 
    & C++ & \yesy & \yesy & \yesy & \yesy & \non & General \\ \hline
    Tensor C++ Library~\cite{tensorCPP} 			
    & C++ & \yesy & \yesy & \yesy & \yesy & \non & General \\ \hline
    HUJI Tensor Library~\cite{huji} 		
    & C++ & \yesy & \yesy & \yesy & \yesy & \non & General \\ \hline
    TCE~\cite{TCE}
    & DSL & \yesy & \yesy & \yesy & \yesy & \non & Chemistry \\ \hline
    Cyclops~\cite{CTF}
    & C++ & \yesy & \yesy & \yesy & \yesy & \non & General \\ \hline
    Tensor Toolbox~\cite{TensorToolbox}
    & MATLAB & \yesy & \yesy & \yesy & \yesy & \non & General \\ \hline
    N-way Toolbox~\cite{Nway-Paper,Nway}
    & MATLAB & \yesy & \yesy & \yesy & \yesy & \non & Chemistry \\ \hline
    CuBatch~\cite{CuBatch}
    & MATLAB & \yesy & \yesy & \yesy & \yesy & \non & ML \\ \hline
    PLS-Toolbox~\cite{PLS-toolbox}
    & MATLAB & \yesy & \yesy & \non & \yesy & \non & Chemistry \\ \hline
    TDALAB~\cite{TDALAB,TDALAB_online}
    & MATLAB & \yesy & \yesy & \yesy & \yesy & \non & General \\ \hline
    TENSORBOX~\cite{TENSORBOX}
    & MATLAB & \yesy & \yesy & \yesy & \yesy & \non & General \\ \hline
    Tensorlab~\cite{Tensorlab}
    & MATLAB & \yesy & \yesy & \yesy & \non & \non & General \\ \hline
    PARCUBE~\cite{PARCUBE}
    & MATLAB & \yesy & \yesy & \yesy & \non & \non & General \\ \hline
    Tensor Train Toolbox~\cite{tt-toolbox}
    & MATLAB & \yesy & \yesy & \yesy & \non & TT & General \\ \hline
    Hier. Tucker Toolbox~\cite{HT,Kressner:2014:A9H:2610268.2538688} 		& MATLAB & \yesy & \yesy & \yesy & \non & HT & General \\ \hline
    TensorCalculus~\cite{Calculus}
    & C++ & \yesy & \yesy & \yesy & \yesy & \yesy & General \\ \hline
    Tensor Network Theory~\cite{TNT}
    & C++ & \yesy & \yesy & \yesy & \yesy & \yesy & Physics \\
    \hline
    \end{tabular}
\end{center}
%=======================================================================
\section{Conclusions} \label{sec:conclusions}
%=======================================================================


%=======================================================================
%\section{Acknowledgments} \label{sec:acknowledgments}
%=======================================================================

\bibliographystyle{abbrv} \bibliography{tensbib}

\clearpage\newpage
\appendix

% eof


%%%%%%%%%%%%%%%%%%%%%%%%%%%%%%%%%%%%%%%%%%%%%%%%%%%%%%%%%%%%%%%%%%%%%%%%
\end{document}
%%%%%%%%%%%%%%%%%%%%%%%%%%%%%%%%%%%%%%%%%%%%%%%%%%%%%%%%%%%%%%%%%%%%%%%%

%eof
